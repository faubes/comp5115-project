% Math and Comp Sci Student
% Homework template by Joel Faubert
% Ottawa University
% Fall 2019
%
% This template uses packages fancyhdr, multicol, esvect and amsmath
% http://ctan.cms.math.ca/tex-archive/macros/latex/contrib/esvect/esvect.pdf
% http://www.ams.org/publications/authors/tex/amslatex
%
% These packages are freely distribxted for use but not
% modification under the LaTeX Project Public License
% http://www.latex-project.org/lppl.txt

\documentclass[letterpaper, 10pt]{article}
% \usepackage[text={8in,10in},centering, margin=1in,headheight=28pt]{geometry}
\usepackage[margin=1in, centering, headheight=28pt]{geometry}
\usepackage{fancyhdr}
\usepackage{esvect}
\usepackage{amsmath}
\usepackage{bbold}
\usepackage{amsfonts}
\usepackage{amssymb}
\usepackage{amsthm}
\usepackage{mathrsfs}
\usepackage{mathtools}
\usepackage{multicol}
\usepackage{enumitem}
\usepackage{verbatimbox}
\usepackage{fancyvrb}
\usepackage{hyperref}
\usepackage[pdftex]{graphicx}
\graphicspath{ {./img/} }
\usepackage[space]{grffile}
\usepackage{bm}
\usepackage{float}
\usepackage{listings} % for code
%\usepackage{minted}

\lstset{language=Matlab}
%\usepackage{sxbfigxre}

% Configure margins
\pagestyle{fancy}
% \hoffset -0.75pt
% \voffset -0.8pt
% \oddsidemargin 0pt
% \topmargin 0pt
% \headheight 25pt
% \headsep 20pt
% \textheight 8.25in
% \textwidth 6.25 in
% \marginparsep 5pt
% \marginparwidth 0.5in
% \footskip 10pt
% \marginparpush 0pt
\paperwidth 8.5in
\paperheight 11in

% Configxre headers and footers

\lhead{COMP5115 \\ Prof. Oliver Von Kaick }
\rhead{Jo\"el Faubert \\ University of Ottawa \# 2560106}
\chead{Project Proposal - Poisson Surface reconstruction \\ 15-11-2019}
\rfoot{\today}
\fancyhfoffset[l]{40pt}
\fancyhfoffset[r]{40pt}
\renewcommand{\headrulewidth}{0.4pt}
\renewcommand{\footrulewidth}{0.4pt}
\setlength{\parskip}{10pt}
\setlist[enumerate]{parsep=10pt, itemsep=10pt}

% Define shortcuts

\newcommand{\floor}[1]{\lfloor #1 \rfloor}
\newcommand{\ceil}[1]{\lceil #1 \rcleil}

% matrices
\newcommand{\bpm}{\begin{bmatrix}}
\newcommand{\epm}{\end{bmatrix}}
\newcommand{\vm}[3]{\begin{bmatrix}#1\\#2\\#3\end{bmatrix}}
\newcommand{\Dmnt}[9]{\begin{vmatrix}#1 & #2 & #3 \\ #4 & #5 & #6 \\ #7 & #8 & #9 \end{vmatrix}}
\newcommand{\dmnt}[4]{\begin{vmatrix}#1 & #2 \\ #3 & #4 \end{vmatrix}}
\newcommand{\mat}[4]{\begin{bmatrix}#1 & #2\\#3 & #4\end{bmatrix}}

% common sets
\newcommand{\R}{\mathbb{R}}
\newcommand{\Qu}{\mathbb{Q}}
\newcommand{\Na}{\mathbb{N}}
\newcommand{\Z}{\mathbb{Z}}
\newcommand{\Rel}{\mathcal{R}}
\newcommand{\F}{\mathcal{F}}
\newcommand{\U}{\mathcal{U}}
\newcommand{\V}{\mathcal{V}}
\newcommand{\K}{\mathcal{K}}
\newcommand{\M}{\mathcal{M}}

% Power set
\newcommand{\PU}{\mathcal{P}(\mathcal{U})}

%norm shortcut
\DeclarePairedDelimiter{\norm}{\lVert}{\rVert}

% projection, vectors
\DeclareMathOperator{\proj}{Proj}
\newcommand{\vctproj}[2][]{\proj_{\vv{#1}}\vv{#2}}
\newcommand{\dotprod}[2]{\vv{#1}\cdot\vv{#2}}
\newcommand{\uvec}[1]{\boldsymbol{\hat{\textbf{#1}}}}

% derivative
\def\D{\mathrm{d}}

% big O
\newcommand{\bigO}{\mathcal{O}}

% probability
\newcommand{\Expected}{\mathrm{E}}
\newcommand{\Var}{\mathrm{Var}}
\newcommand{\Cov}{\mathrm{Cov}}
\newcommand{\Entropy}{\mathrm{H}}
\newcommand{\KL}{\mathrm{KL}}

\DeclareMathOperator*{\argmax}{arg\,max}
\DeclareMathOperator*{\argmin}{arg\,min}

\setlist[enumerate]{itemsep=10pt, partopsep=5pt, parsep=10pt}
\setlist[itemize]{itemsep=5pt, partopsep=5pt, parsep=10pt}

\begin{document}

\section{Objective}
\subsection{Context}

The Intel RealSense 2 can capture colour point clouds in .ply and .bag files.
The goal is to use this data to reconstruct high quality models in spite
of the noisy input data.

Poisson surface reconstruction \cite{kazhdan2006poisson} may be suited for
quality offline reconstruction.

\subsection{Related work}

Radial basis function reconstruction \cite{carr2001reconstruction} is another technique for
point cloud reconstruction.

Another class of techniques use Voronoi diagrams \cite{amenta1998new} and similar
graph construction-based techniques.

\subsection{Selected approach}

Poisson surface reconstruction uses the gradient of the normal field to solve
for an indicator function. The indicator function can then be used to extract
a surface.

\section{Methodology}

I frankly have no idea how Poisson reconstruction works yet.

\section{Implementation}

\subsection{Programming Languages/Libraries/Data Structures}

\begin{itemize}
  \item Intel RealSense SDK
  \item Point Cloud Library, or possibly
  \item CGAL
\end{itemize}

\subsection{Data sets}
\begin{itemize}
  \item The Stanford Bunny
  \item Data acquired from the RGB-D Camera
\end{itemize}

\section{Timeline}

So far, I've toyed with a few examples bundled with the RealSense SDK.
I am able to take snapshots and ``depth videos'' using the camera.

\begin{enumerate}
  \item Study the paper on Poisson Reconstruction and the provided implementation (2-3 days)
  \item Come up with a way to try to improve accuracy -- either by eliminating noise or detecting and preserving features. Possibly use multiple input frames to smooth readings, or make use of colour information. (1-2 days)
  \item Implement and test (2 weeks)
\end{enumerate}
\appendix
\newpage
\bibliographystyle{IEEEtran}
\bibliography{references}

\end{document}
