\begin{frame}
%\frametitle{Title}
\begin{center}
{\Large
3D Reconstruction from Depth Images\\
COMP5115 Project \\
Fall 2019 }
\end{center}
\end{frame}

\begin{frame}
\frametitle{Outline}
\begin{itemize}
\item Introduction and Motivation
\item Related Works
\item Results and Conclusion
\end{itemize}
\end{frame}

\begin{frame}
\frametitle{Introduction and Motivation}
\begin{itemize}
\item Purchased an Intel RealSense D435 Camera.
\item Studied STAR paper by Zollh\"ofer to see how it could be used.
\item Installing/compiling IntelRealSense SDK with OpenCV and PCL
\end{itemize}

\end{frame}

\begin{frame}
\frametitle{Problem Statement}
Problem: process a stream of RGB-D frames for model reconstruction
\begin{itemize}
  \item Tracking: estimate the pose (position + orientation) of the camera.
  Camera presumed moving through space -- need to keep track of position and which way it's pointing.
  \item Mapping: (incrementally) build a model of the scene captured by camera.
\end{itemize}
\end{frame}


\begin{frame}[allowframebreaks]
  \frametitle<presentation>{Related Works}
  \begin{thebibliography}{1}
  \beamertemplatearticlebibitems
  \bibitem{Keselman2017Realsense}
  \newblock Keselman, Leonid, et al.
  \newblock Intel realsense stereoscopic depth cameras.
  \newblock IEEE Conference on Computer Vision and Pattern Recognition Workshops. 2017.
  \bibitem{Cignoni2008Meshlab}
  \newblock Cignoni, Paolo, et al.
  \newblock Meshlab: an open-source mesh processing tool.
  \newblock Eurographics Italian chapter conference. Vol. 2008.
  \bibitem{Pagliari2014KinectFusion}
  \newblock Pagliari, Diana, et al.
  \newblock Kinect Fusion improvement using depth camera calibration.
  \newblock The International Archives of Photogrammetry, Remote Sensing and Spatial Information Sciences 40.5 (2014): 479.
\end{thebibliography}
\end{frame}
